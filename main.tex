%----------------------------------------------------------------------------------------
%	PREÁMBULO
%----------------------------------------------------------------------------------------

\documentclass[11pt, oneside]{book}
\usepackage[paperwidth=17cm, paperheight=22.5cm, bottom=2.5cm, right=2.5cm]{geometry}

% El borde inferior puede parecerles muy amplio a la vista. Les recomiendo hacer una prueba de impresión antes para ajustarlo

\usepackage{amssymb,amsmath,amsthm} % Símbolos matemáticos
\usepackage[spanish]{babel}
\usepackage[utf8]{inputenc} % Acentos y otros símbolos 
\usepackage{enumerate}
\usepackage{hyperref} % Hipervínculos en el índice
\usepackage{graphicx}
%\usepackage{subfig} % Subfiguras
\graphicspath{{Imagenes/}} % En qué carpeta están las imágenes

% Para eliminar guiones y justificar texto
\tolerance=1
\emergencystretch=\maxdimen
\hyphenpenalty=10000
\hbadness=10000

\linespread{1.25} % Asemeja el interlineado 1.5 de Word

\let\oldfootnote\footnote % Deja espacio entre el número del pie de página y el inicio del texto
\renewcommand\footnote[1]{%
\oldfootnote{\hspace{0.05mm}#1}}

\renewcommand{\thefootnote} {\textcolor{Black}{\arabic{footnote}}} % Súperindice a color negro

\setlength{\footnotesep}{0.75\baselineskip} % Espaciado entre notas al pie

\usepackage{fnpos} % Footnotes al final de pág.

\usepackage[justification=centering, font=bf, labelsep=period, skip=5pt]{caption} % Centrar captions de tablas y ponerlas en negritas

\newcommand{\imagesource}[1]{{\footnotesize Fuente: #1}}

\usepackage{tabularx} % Big tables
\usepackage{graphicx}
\usepackage{adjustbox}
\usepackage{longtable}

\usepackage{float} % Float tables

\usepackage[usenames,dvipsnames]{xcolor} % Color

\usepackage{pgfplots} % Gráficas
\pgfplotsset{compat=newest}
\pgfplotsset{width=7.5cm}
\pgfkeys{/pgf/number format/1000 sep={}}

\begin{document}

%----------------------------------------------------------------------------------------
%	PORTADA
%----------------------------------------------------------------------------------------

\title{Confianza institucional, inclusión social, intensidad religiosa y percepción tecnológica como factores de innovación para el crecimiento económico} % Con este nombre se guardará el proyecto en writeLaTex

\begin{titlepage}
\begin{center}

\textsc{\Large Instituto Tecnológico Autónomo de México}\\[2em]

%Figura
\begin{figure}[h]
\begin{center}
\includegraphics[scale=0.50]{itam_logo.png}
\end{center}
\end{figure}

% Pueden modificar el tamaño del logo cambiando la escala

\textbf{\LARGE Confianza institucional, inclusión social, intensidad religiosa y percepción tecnológica como factores de innovación para el crecimiento económico}\\[2em]

\textsc{\large Tesis}\\[1em]

\textsc{\large que para obtener el título de}\\[1em]

\textsc{\LARGE Licenciado en Economía / Relaciones Internacionales}\\[1em]

\textsc{\large Presenta}\\[1em]

\textsc{\LARGE Farid Hannan Goyri}\\[1em]

\textsc{\large Asesora}\\[1em]

\textsc{\LARGE Dr. Raúl Livas Elizondo}\\[2em]

% Asegúrense de escribir el nombre completo de su asesor

\end{center}

\vspace*{\fill}
\textsc{Ciudad de México \hspace*{\fill} 2018}

\end{titlepage}

%----------------------------------------------------------------------------------------
%	DECLARACIÓN
%----------------------------------------------------------------------------------------

\thispagestyle{empty}

\vspace*{\fill}
\begingroup

\noindent
«Con fundamento en los artículos 21 y 27 de la Ley Federal del Derecho de Autor y como titular de los derechos moral y patrimonial de la obra titulada ``\textbf{Confianza institucional, inclusión social, intensidad religiosa y percepción tecnológica como factores de innovación para el crecimiento económico}'', otorgo de manera gratuita y permanente al Instituto Tecnológico Autónomo de México y a la Biblioteca Raúl Bailléres Jr., la autorización para que fijen la obra en cualquier medio, incluido el electrónico, y la divulguen entre sus usuarios, profesores, estudiantes o terceras personas, sin que pueda percibir por tal divulgación una contraprestación.»

% Asegúrense de cambiar el título de su tesis en el párrafo anterior

\centering 

\vspace{5em}

\rule[1em]{20em}{0.5pt} % Línea para la fecha

\textsc{Fecha}
 
\vspace{8em}

\rule[1em]{20em}{0.5pt} % Línea para la firma

\textsc{Farid Hannan Goyri}

\endgroup
\vspace*{\fill}

%----------------------------------------------------------------------------------------
%	DEDICATORIA
%----------------------------------------------------------------------------------------

\pagestyle{plain}
\frontmatter

\chapter*{}
\begin{flushright}
\textit{A mis padres,\\ por su incansable esfuerzo.}
\end{flushright}

%----------------------------------------------------------------------------------------
%	AGRADECIMIENTOS
%----------------------------------------------------------------------------------------

\chapter*{Agradecimientos}

\noindent Lorem ipsum dolor sit amet, consectetur adipiscing elit.

% Esta sección es lo único que la gente lee. True story :)

%----------------------------------------------------------------------------------------
%	RESUMEN
%----------------------------------------------------------------------------------------

\chapter*{Resumen}

\noindent El modelo de Solow afirma que el crecimiento de largo plazo se origina por un aumento en la productividad total de los factores. Romer agrega que esta productividad es generada por nuevas ideas. Las causas de la innovación son múltiples y en esta investigación se busca elucidar algunas de ellas. Desde una perspectiva heterodoxa que incluye resultados de la neurociencia o la genética, se argumenta que la confianza institucional, la inclusión social y la optimista percepción tecnológica tienen una relación positiva y significativa con la innovación. Entre las posibles razones causales está el hecho de que la confianza incentiva a los emprendedores por arriesgar, que la inclusión no limita el número de ideas potenciales en el mercado y que la optimista percepción es parte de las sociedades dinámicas que buscan innovar. Por el contrario, la intensidad religiosa guarda una relación negativa por la ausencia de actividad creativa que puede suscitar la interacción entre instituciones. La hipótesis se sostiene con un análisis econométrico de cincuenta y cinco países. La tesis cierra con el caso de estudio mexicano, en donde se propone que la etiología del lento crecimiento es la falta de innovación, la cual puede ser explicada por una baja confianza institucional y una alta religiosidad. El análisis está inmerso en la nueva globalización y la cuarta revolución tecnológica.

\pagestyle{plain}

\noindent 

%----------------------------------------------------------------------------------------
%	Summary
%----------------------------------------------------------------------------------------

\chapter*{Summary}

\noindent The Solow model states that long-term growth is caused by an increase in total factor productivity. Romer adds that this productivity is generated by new ideas. The causes of innovation are multiple and this research seeks to elucidate some of them. From a heterodox perspective that includes results from neuroscience or genetics, it is argued that institutional confidence, social inclusion and an optimistic technological perception have a positive and significant relationship with innovation. Among the possible causal reasons is the fact that confidence encourages entrepreneurs to take risks, that inclusion does not limit the number of potential ideas in the market and that an optimistic perception is part of the dynamic societies that innovate. On the contrary, religious intensity has a negative relationship due to the absence of creative activity that can arouse due to the interaction between institutions. The hypothesis is supported by an econometric analysis of fifty-five countries. The thesis closes with a case study on Mexico, where it is proposed that the etiology of its slow growth is the lack of innovation, which can be explained by low institutional confidence and high religiosity. The analysis is immersed in the new globalization and the fourth technological revolution.

\pagestyle{plain}

\noindent 

%----------------------------------------------------------------------------------------
%	TABLA DE CONTENIDOS
%---------------------------------------------------------------------------------------

\tableofcontents

%----------------------------------------------------------------------------------------
%	ÍNDICE DE CUADROS Y FIGURAS
%---------------------------------------------------------------------------------------

\listoftables

\listoffigures

%----------------------------------------------------------------------------------------
%	TESIS
%----------------------------------------------------------------------------------------

\mainmatter % Empieza la numeración de las páginas

\pagestyle{plain}

% Incluye los capítulos en el fólder de capítulos

\include{Chapters/0.Intro}

\include{Chapters/1.Rev}

\include{Chapters/2.Esp}

\chapter{Modelo econométrico}

\noindent Este capítulo tiene el propósito de probar empíricamente las hipótesis planteadas previamente y está dividido en dos secciones.

\newpage

\section{Regresión lineal múltiple}

\subsection{Modelo y resultados}

\noindent Ulteriormente, se corrió una regresión lineal múltiple en la que la variable dependiente es la innovación y las variables independientes son los cuatro índices construidos: 
 
\begin{equation}
   Y = \beta_{0} + \beta_{1} X_{i1} + \beta_{2} X_{i2} + \beta_{3} X_{i3} + \beta_{4} X_{i4} +\varepsilon, \label{1}
\end{equation}

% Para escribir ecuaciones

\begin{table}[H]
\centering
\caption{Resultados del modelo econométrico}
\label{REG}
\begin{tabular}{|ccccc|}
  \hline
 & Estimador & Desv. est. & Valor $t$ & Pr($>|t|$) \\ 
  \hline
  $\alpha$ & x & x & x & x \\ 
  CI & x & x & x & x \\ 
  IS & x & x & x & x \\ 
  IR & x & x & x & x \\ 
  PT & x & x & x & x \\ 
   \hline
\multicolumn{5}{|c|}{Error est. de res. = x con x gr. de libertad}
    \\
    \multicolumn{5}{|c|}{$R^2$ = x y $\bar{R}^2$ = x}
 \\
 \multicolumn{5}{|c|}{Estadístico $F$ = x, con un valor $p$ = x }
      \\
\hline
\end{tabular}

\end{table}


\include{Chapters/4.Mex}

\include{Chapters/5.Conc}


%----------------------------------------------------------------------------------------
%	APÉNDICES
%----------------------------------------------------------------------------------------

\begin{appendix}

\include{Apendices/ApA}

\end{appendix}

%----------------------------------------------------------------------------------------
%	BIBLIOGRAFÍA
%----------------------------------------------------------------------------------------


\chapter*{Referencias}
\addcontentsline{toc}{chapter}{Referencias}

% Macro. Esto es muy importante, no lo borren

\makeatletter
\renewenvironment{thebibliography}[1]
     {\@mkboth{\MakeUppercase\refname}{\MakeUppercase\refname}%
      \list{}%
           {\setlength{\labelwidth}{0pt}%
            \setlength{\labelsep}{0pt}%
            \setlength{\leftmargin}{\parindent}%
            \setlength{\itemindent}{-\parindent}%
            \@openbib@code
            \usecounter{enumiv}}%
      \sloppy
      \clubpenalty4000
      \@clubpenalty \clubpenalty
      \widowpenalty4000%
      \sfcode`\.\@m}
     {\def\@noitemerr
       {\@latex@warning{Empty `thebibliography' environment}}%
      \endlist}
\makeatother

\begin{thebibliography}{111}

% Lista

% La manera recomendada para citar papers o libros en el formato de Chicago esta en el siguiente vínculo: https://www.chicagomanualofstyle.org/tools_citationguide/citation-guide-2.html

% Es importante poner el apellido del autor seguido del año de publicación, una coma y las páginas consultadas en el texto antes de puntuar y entre paréntesis para las citas en el cuerpo de la tesis

% Ejemplo:

% Las \textit{causas próximas} del crecimiento son conocidas: tecnología, capital humano y físico. La pregunta es ¿por qué unos países sí tienen estas causas próximas y otros no? La respuesta son las \textit{causas fundamentales:} suerte, geografía, cultura e instituciones (Acemoglu 2009, 110).

%AAAAA
\bibitem{abram57} Abramovitz, Moses. 1957. «Resources on Output Trends in the United States since 1870.» \textit{The American Economic Review} 46 (2): 5–23.

\bibitem{acemoglu09} Acemoglu, Daron. 2009. \textit{Introduction to Modern Economic Growth.} Princeton: Princeton University Press.

%BBBBB

%CCCCC

%DDDDD

%EEEEE

%FFFFF

%GGGGG

%HHHHH

%IIIII

%JJJJJ

%KKKKK

%LLLLL

%MMMMM

%NNNNN

%OOOOO

%PPPPP

%QQQQQ

%RRRRR

%SSSSS

%TTTTT

%UUUUU

%VVVVV

%WWWWW

%XXXXX

%YYYYY

%ZZZZZ

\end{thebibliography}

\newpage
\thispagestyle{empty}
\begin{table}[p]
\centering
\small
\label{ed}
\begin{tabular}{c}
\textit{Confianza institucional, inclusión social,}\\ \textit{intensidad religiosa y percepción}\\ \textit{tecnológica como factores de innovación}\\ \textit{para el crecimiento económico,}\\ escrito por Farid Hannan,\\ se terminó de imprimir en diciembre de 2018\\ en los talleres de Tesis Matozo.\\ Campeche 156, colonia Roma,\\ Ciudad de México.
\end{tabular}
\end{table}

% Si lo prefieren, avisen a su taller que esta página ya la incluyeron ustedes para que no les impriman las que ellos usan. Lo recomiendo ampliamente

%%%%%%%%%%%%%%%%%%%%%%%%%%%%%%%%%%%%%%%%%%%%%%%%%%%%%%%%%%%%%%%%%%%%%%%%%%%%%%%%%%%%%%%

\end{document}