\chapter{Modelo econométrico}

\noindent Este capítulo tiene el propósito de probar empíricamente las hipótesis planteadas previamente y está dividido en dos secciones.

\newpage

\section{Regresión lineal múltiple}

\subsection{Modelo y resultados}

\noindent Ulteriormente, se corrió una regresión lineal múltiple en la que la variable dependiente es la innovación y las variables independientes son los cuatro índices construidos: 
 
\begin{equation}
   Y = \beta_{0} + \beta_{1} X_{i1} + \beta_{2} X_{i2} + \beta_{3} X_{i3} + \beta_{4} X_{i4} +\varepsilon, \label{1}
\end{equation}

% Para escribir ecuaciones

\begin{table}[H]
\centering
\caption{Resultados del modelo econométrico}
\label{REG}
\begin{tabular}{|ccccc|}
  \hline
 & Estimador & Desv. est. & Valor $t$ & Pr($>|t|$) \\ 
  \hline
  $\alpha$ & x & x & x & x \\ 
  CI & x & x & x & x \\ 
  IS & x & x & x & x \\ 
  IR & x & x & x & x \\ 
  PT & x & x & x & x \\ 
   \hline
\multicolumn{5}{|c|}{Error est. de res. = x con x gr. de libertad}
    \\
    \multicolumn{5}{|c|}{$R^2$ = x y $\bar{R}^2$ = x}
 \\
 \multicolumn{5}{|c|}{Estadístico $F$ = x, con un valor $p$ = x }
      \\
\hline
\end{tabular}

\end{table}
